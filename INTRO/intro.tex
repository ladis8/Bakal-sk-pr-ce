%!TEX ROOT=ctutest.tex
\chapter{Úvod}

    Funkčnost a spolehlivost technických zařízení se výrazně podepisuje na provozních nákladech a bezpečnosti moderních systémů v průmyslu po celou dobu jejich fungování. 
    Za pomoci diagnostiky stavu zařízení a následné údržby je cílem každé firmy minimalizovat ztráty a rozsah škod způsobených jejich poruchami nebo úplným selháním. Tato diagnostika stavu a údržba zařízení je prováděna  všude kolem nás, ve všech technologických oblastech, od těžkých strojních přístrojů (přístavní jeřáby, důlní zařízení), po leteckou techniku (letadla, helikoptéry) až po jemná biotechnologická zařízení, jako jsou např. kardiostimulátory.
    Na otázku, jak efektivně vyřešit poruchy nejrůznějších strojů, existuje jednoduchá odpověď - předejít jejím vznikům. Tato myšlenka nese filosofii konceptů Condition Based Maintenence (CBM) a Predictive Maintenence, které jsou jedním ze základních stavebních kamenů moderní diagnostiky zařízení v průmyslu.
    
    Preventivní predeterminovaná údržba s pravidelnými kontroly a diagnostikou stavu zařízení byla po dlouhou dobu v minulosti nejspolehlivější metodou, jak předejít poruše stroje. V dnešní době, kdy je ale kladen důraz na maximální efektivnost, nese tento koncept údržby mnohá úskalí. Pravidelné kontroly zvyšují finanční náklady na provoz, mrhají jak lidskými tak materiálními zdroji, často musí kvůli nim být pozastavena výroba a i přes to neeliminují riziko poruchy.
    
    Prediktivní údržba podle technického stavu (CBM) díky moderním technologiím dnešní doby, které umožňují monitorovat a diagnostikovat stav zařízení v reálném čase, dokáže předvídat poruchu ještě před tím, než k ní dojde, a rozhoduje tak o provedení údržby v tu chvíli, kdy ji je opravdu třeba. Tento koncept je tedy proto v mnoha aplikacích výhodnější, přináší časové, pracovní a materiálové úspory a je velice vhodný zejména pro zařízení, jejichž selhání má fatální následky.

\subsection{Diagnostika stavu rotačních zařízení}
    K selhání motorů může dojít mnoha způsoby, ačkoliv fakta ukazují, že 40 až 90 \% poruch je zapříčiněno defektem ložisek. Tyto poruchy navíc často způsobují selhání celého motoru, a proto se jejich diagnostika a  sledování stavu stává důležitým úkolem. 
    Pro monitorování stavu ložisek se nejčastěji používá vibrační analýza a měření teploty. Tyto veličiny reflektují stav ložisek, jejich mechanické namáhání, opotřebení materiálů, vznikající trhliny a tvarové nedokonalosti.
    Z vlastností vibračního signálu lze identifikovat příznaky přicházející poruchy ložisek, avšak tento úkol není vůbec snadný. Pro přesnou expertízu je třeba vědět, které charakteristiky vibračního signálu je třeba sledovat a jakým způsobem jsou spojeny s vlastnostmi ložiska.     
    

\subsection{Monitorování zařízení v rámci IIoT}
    V dnešních dnech je fenomén Internet of Things (IoT) česky Internet věcí jedním z nejcitovanějších témat v technologicky zaměřených médiích. Pro mnohé se ale tento pojem stal symbolem své pouze jedné větve - spotřebitelského internetu věcí, který je z popularizačního hlediska zajímavější, zaměřen na chytrá města a domácnosti.\\
    Mnohem větší význam, ale nese Industrial Internet of Things (IIoT) česky Průmyslový internet věcí, často označovaný jako Průmysl 4.0. IIoT v první řadě poskytuje lepší přehled o aktivitách v průmyslové výrobě a to prostřednictvím monitorujících senzorů, konektivity a cloud computingu.
    Díky zpětné vazbě získané z datové analýzy umožňuje transformovat a především optimalizovat výrobní operace, což zvyšuje zejména produktivitu, efektivnost a náklady.\\
    IIoT je tedy klíčovým faktorem umožňující realizovat prediktivní údržbu a CBM.\\
    
    

\subsection{LPWAN a vibrační analýza}
    
    Bezdrátový IIoT představuje další větev Průmyslového Internetu věcí. Jeho využití, ale přináší řadu úskalí jako je dovolená šířka pásma, elektromagnetická kompatibilita (EMC), spolehlivost komunikace a výdrž na baterii, která se výrazně snižuje díky použití bezdrátových přijímačů a vysílačů. Obecně lze říci, že díky těmto důvodům je pro průmyslové aplikace vždy vhodnější využití drátové komunikace nežli bezdrátové. 
    Většina stávajících řešení poskytujících vibrační analýzu využívá proto pro komunikaci ve výrobních halách již dobře dostupná rozhraní jako Ethernet Powerlink nebo HART (Highway Addressable Remote Transducer Protocol).\\
    V mnoha případech je ale použití drátových technologií problematické. Často jsou také opomíjena rotační zařízení například v elektrárnách, tunelech nebo na rozlehlých stavebních plochách, kde by vytvoření drátové sítě bylo velmi nákladné, či úplně nemožné a jejichž monitorování by tak nebylo možné. V těchto případech je tedy nasnadě využití bezdrátových komunikačních sítí LPWAN (Low Power Wide Area Network) jako například LoRa, SigFox či NB-IoT. Tyto sítě se vyznačují hvězdicovitou architekturou, kdy koncová zařízení odesílají data do bran, které jsou připojena k internetu na cloudovou platformu.  

\subsection{Cíl práce}
    
    Cílem této práce bylo navrhnout a realizovat komplexní pokročilý systém pro monitorování stavu průmyslových zařízení pomocí pokročilé analýzy vibrací a teploty.\\
    Sytém se bude odlišovat od existujících zařízení na trhu zejména svoji konfigurovatelností. Uživatel bude moci pohodlně ve webovém prostředí měnit parametry monitorovací jednotky, díky čemuž bude moci přizpůsobit zpracování monitorovaných veličin a dosáhnout tak různorodějších a pokročilejších analýz.\\
    
    Důraz bude kladen zejména na cenu, dostupnost a univerzální použití. Jako komunikační médium se bude využívat LPWAN síť LoRa s velkým dosahem, ale nízkým datovým tokem, což se může na první pohled zdát jako nesmyslný krok. Tato technologie ale umožní sledovat stav rotačních zařízení nejen v chytrých průmyslových výrobnách ale i na rozsáhlých plochách pro velké vzdálenosti, což odliší systém od existujících řešení a umožní jej používat i v prostředích, kde nasazení podobného systému bylo komplikované jako například větrné elektrárny, tunely nebo rozlehlé stavební plochy.\\
    
    Mimo jiné bude pozornost věnována flexibilitě celého systému prostřednictvím použití databázového modelu a RESTového serveru, které umožní poskytnutí naměřených dat aplikacím třetích stran, které zákazníci budou moci zpracovávat podle svých požadavků ve vlastních systémech.
    
Zdroje

https://www.bozpinfo.cz/josra/prediktivni-udrzba-metody-technicke-prognostiky-seznameni-se-s-problematikou

Industry 4.0: The Industrial Internet of Things

\question{TYPOGRAFIE: prvni odstavec}
